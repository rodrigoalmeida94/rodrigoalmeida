\documentclass[a4paper,9pt]{article}
\usepackage{fancyhdr}
\usepackage[left=0.3in,top=0.25in,right=0.3in,bottom=0.3in,footskip=0.15in]{geometry}
\usepackage{hyperref}

% Metadata
\hypersetup{
    pdftitle={Rodrigo Almeida - CV},
    pdfauthor={Rodrigo Almeida},
    pdfsubject={Curriculum Vitae},
    pdfkeywords={Geo-Information, AI, Cloud Engineer, CV, Weather, Climate},
    pdfcreator={LaTeX with Hyperref},
    colorlinks=true,
    linkcolor=linkcolor,
    urlcolor=linkcolor,
}

% Define timestamp command with default if not provided
\providecommand{\cvtimestamp}{\today}

% Footer setup
\pagestyle{fancy}
\fancyhf{} % Clear all header/footer fields
\renewcommand{\headrulewidth}{0pt}
\renewcommand{\footrulewidth}{0pt}
\rfoot{\tiny\color{gray} Generated on \cvtimestamp}
\usepackage{titlesec}
\usepackage{enumitem}
\usepackage{fontawesome5}
\usepackage{xcolor}
\usepackage[T1]{fontenc} % Required for proper font selection
\usepackage{inconsolata}

% Colors matches typical terminal themes
\definecolor{promptcolor}{RGB}{87, 199, 255} % Light Blue for prompt
\definecolor{commandcolor}{RGB}{255, 121, 198} % Pink for command
\definecolor{headercolor}{RGB}{0, 100, 0} % Dark green
\definecolor{linkcolor}{RGB}{64, 64, 64} % Dark Grey for links

% Set default font family to typewriter
\renewcommand{\familydefault}{\ttdefault}
\renewcommand{\itdefault}{sl} % Inconsolata lacks true italics, use slanted instead
\small

% Section formatting - Terminal Prompt Style
\titleformat{\section}{\large\bfseries\color{black}}{}{0em}{}[\titlerule]
\titlespacing{\section}{0pt}{4pt}{6pt}

% Custom commands
\newcommand{\entry}[4]{
  \noindent\textbf{#1} \hfill #2 \\
  \textsl{#3} \hfill \textsl{#4}
}

\begin{document}
% \pagestyle{empty} % Removed to allow fancyhdr footer

% Header
\begin{center}
    {\Huge \textbf{Rodrigo Almeida}} \\ [2pt]
    {\large Geo-Information, AI Researcher, and Cloud Engineer} \\ [2pt]
    \small
    \faMapMarker* \ Berlin, Germany \quad | \quad
    \href{mailto:rodrigo.almeida94@outlook.pt}{\faEnvelope \ rodrigo.almeida94@outlook.pt} \quad | \quad
    \href{https://rodrigoalmeida.me}{\faGlobe} \quad | \quad
    \href{https://github.com/rodrigoalmeida94}{\faGithub} \quad | \quad
    \href{https://linkedin.com/in/rodrigoalmeida94}{\faLinkedin}
\end{center}

% Summary
\section*{\textcolor{promptcolor}{$\sim$} \textcolor{commandcolor}{\$ cat summary.txt}}
\noindent Focused on using AI to understand, predict, and adapt to a changing climate. Experienced in building scalable pipelines for weather data, training geospatial Deep Learning models, and leading engineering teams to solve complex, high-impact problems.

% Publications & Open Source (Moved up)
\noindent
\begin{minipage}[t]{0.48\textwidth}
    \section*{\textcolor{promptcolor}{$\sim$/publications} \textcolor{commandcolor}{\$ ls *.pdf}}
    \begin{itemize}[leftmargin=*, noitemsep, topsep=0pt]
        \item \href{https://doi.org/10.48550/arXiv.2511.17176}{Predictive Skill of AI Weather Models for Extreme Events using Uncertainty Quantification} \textsl{(Pre-print)}
        \item \href{https://doi.org/10.1007/s13580-020-00316-9}{Inferring Ethylene Distribution in Apple Orchard: A Pilot Study for Optimal Sampling} \textsl{(Journal)}
        \item \href{https://doi.org/10.3390/s19020372}{Potential Application of Flying Ethylene-Sensitive Sensors for Ripeness Detection} \textsl{(Journal)}
        \item \href{https://doi.org/10.48550/arXiv.2002.00580}{Super-resolution of multispectral satellite images using convolutional neural networks} \textsl{(Pre-print)}
    \end{itemize}
\end{minipage}
\hfill
\begin{minipage}[t]{0.48\textwidth}
    \section*{\textcolor{promptcolor}{$\sim$/open-source} \textcolor{commandcolor}{\$ git log --oneline}}
    \begin{itemize}[leftmargin=*, noitemsep, topsep=0pt]
        \item \href{https://github.com/brightbandtech/ExtremeWeatherBench/pull/284}{\textbf{brightbandtech/ExtremeWeatherBench}: Add ROCSS metric}
        \item \href{https://github.com/nvidia/earth2studio/pull/500}{\textbf{NVIDIA/earth2studio}: Add AIFS ENS}
        \item \href{https://github.com/nvidia/earth2studio/pull/256}{\textbf{NVIDIA/earth2studio}: Add GraphCast}
        \item \href{https://github.com/SkyTruth/cerulean-cloud/pull/60}{\textbf{SkyTruth/cerulean}: Production deploy}
        \item \href{https://github.com/NASA-IMPACT/csdap-cumulus/pull/8}{\textbf{NASA/cumulus}: Switch to AWS Cognito}
        \item \href{https://github.com/cogeotiff/rio-tiler/pull/431}{\textbf{cogeo/rio-tiler}: Use httpx}
        \item \href{https://github.com/up42/up42-py/pull/182}{\textbf{up42/up42-py}: Add CI/CD}
        \item \href{https://github.com/calebrob6/land-cover/pull/4}{\textbf{calebrob6/land-cover}: Evaluation script to Python}
    \end{itemize}
\end{minipage}

% Experience
\section*{\textcolor{promptcolor}{$\sim$/experience} \textcolor{commandcolor}{\$ ls -t}}

\entry{Fraunhofer HHI}{Berlin}{ML Researcher, Applied AI}{Feb 2025 -- Present}
\begin{itemize}[leftmargin=15pt, nosep]
    \item Quantifying uncertainty in global AI weather models, evaluating on extreme events.
    \item Developing Climate and Weather AI applications.
\end{itemize}
\vspace{2pt}

\entry{Jua.ai}{Remote}{Eng. Manager, Data Team}{Mar 2023 -- Jun 2024}
\begin{itemize}[leftmargin=15pt, nosep]
    \item Led a team of 2 engineers and worked closely with product.
    \item Ingested 30 different sources of historical weather observation data into a common data warehouse, using Zarr and Parquet (> 500 TB).
    \item Created live ETL pipelines for weather data using Prefect, deploying it using Pulumi in GCP.
\end{itemize}
\textsl{Senior Data Engineer} \hfill \textsl{Nov 2022 -- Mar 2023}
\begin{itemize}[leftmargin=15pt, nosep]
    \item Using Zarr and Dask, created a pipeline to downscale weather forecasts to 1x1 km at the global level, 4x a day, using a deep learning model.
    \item Developed live ingestion pipelines for multiple weather data sources (reanalysis data and observation data), using AWS Step Functions.
\end{itemize}
\vspace{2pt}

\entry{Development Seed}{Remote}{Cloud Software Engineer}{Aug 2021 -- Oct 2022}
\begin{itemize}[leftmargin=15pt, nosep]
    \item Developed a multi-cloud (AWS and GCP) and cost-efficient cloud infrastructure for running deep learning-based oil slick detection with Sentinel-1 images.
    \item Developed an ingestion pipeline \& search API that is able to handle millions of images and return similarity, at scale.
\end{itemize}
\vspace{2pt}

\entry{UP42 (Airbus)}{Berlin}{Senior Data Science Engineer}{Jan 2021 -- Jul 2021}
\begin{itemize}[leftmargin=15pt, nosep]
    \item Used FastAPI to develop asynchronous microservices to estimate resource consumption of geospatial workflows.
    \item Developed full CI/CD pipeline for dockerized geospatial processing tools, including live and end-to-end tests.
\end{itemize}
\textsl{Data Science Engineer} \hfill \textsl{Sep 2019 -- Dec 2021}
\begin{itemize}[leftmargin=15pt, nosep]
    \item Developed processing chains for geospatial data in Python with Docker.
    \item Built requirements for compatibility service of different geospatial processing chains.
    \item Conceptualised and trained deep learning model for land cover classification with satellite images using TensorFlow.
\end{itemize}
\vspace{2pt}

\entry{Planet}{Berlin}{Pre-Sales Engineer}{Jul 2018 -- Aug 2019}
\begin{itemize}[leftmargin=15pt, nosep]
    \item Technical consultancy for prospective customers.
    \item Developed internal tools for reporting and data visualisation.
\end{itemize}
\vspace{2pt}

% Education
\section*{\textcolor{promptcolor}{$\sim$/education} \textcolor{commandcolor}{\$ cat degrees}}
\noindent \textbf{MSc Geo-Information Science} (\textsl{Cum laude}), Wageningen Univ. \hfill 2016 -- 2019 \\
\noindent \textbf{BSc Agriculture Engineering}, ISA Lisbon Univ. \hfill 2012 -- 2015

% Skills (Moved to bottom)
\section*{\textcolor{promptcolor}{$\sim$} \textcolor{commandcolor}{\$ grep -r "Skills" .}}
\begin{itemize}[leftmargin=*, noitemsep, topsep=0pt]
    \item \textbf{Lang/Frameworks:} Python, FastAPI, PyTorch, Dask, Pulumi, Prefect
    \item \textbf{Cloud/DevOps:} AWS, GCP, Docker, CI/CD, Terraform, Slurm
    \item \textbf{AI/Data:} Xarray, Zarr, Parquet, GDAL, PostGIS, ML, DL, CV
\end{itemize}

\end{document}
